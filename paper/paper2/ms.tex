\documentclass[twocolumn]{aastex631}
\bibliographystyle{aasjournal}

\usepackage{amsmath}
\usepackage{blindtext}
% Temporary packages
\usepackage{mdframed}


\begin{document}
\title{Blas\'e3D}
\shorttitle{\emph{blas\'e3D}}

\author[0000-0002-2290-6810]{Sujay Shankar}
\author[0000-0002-4020-3457]{Michael Gully-Santiago}
\author[0000-0002-4404-0456]{Caroline V. Morley}
\affil{Department of Astronomy, The University of Texas at Austin, 2515 Speedway, Austin, TX 78712, USA}
\shortauthors{Shankar \& Gully-Santiago \& Morley}

\begin{abstract}
    \blindtext
\end{abstract}

\keywords{}


\section{Introduction}
\blindtext

\begin{mdframed}
    \textbf{Figure: Flowchart-with-words}
\end{mdframed}

\section{Emulating the PHOENIX grid}

\begin{mdframed}
    \textbf{The PHOENIX grid subset}

    \textcolor{lightgray}{\blindtext}
\end{mdframed}

\begin{mdframed}
    \textbf{Emulation with blas\'e}

    - Per-grid point optimization procedure
    - Pretrained model caching
    - Output storage

    \textcolor{lightgray}{\blindtext}
\end{mdframed}

\begin{mdframed}
    \textbf{Open Source Availability}

    \textcolor{lightgray}{\blindtext}
\end{mdframed}

\begin{mdframed}
    \textbf{Figure: Line density across the grid heatmap}
\end{mdframed}


\section{Line-by-Line Fundamental Stellar Properties}
\begin{mdframed}
    \textbf{Conceptual Illustration: Faceted Plot (Teff, Logg, Z) -> Line Profiles}
\end{mdframed}

\begin{mdframed}
    \textbf{Line Recognition}
    - Need: Identifying unique lines
    - Anticipated friction points: line centroids drift
    - Strategy: (Pre-shift centers)
    \textcolor{lightgray}{\blindtext}
\end{mdframed}

\begin{mdframed}
    \textbf{Line property bulk trends across the grid}
    \textcolor{lightgray}{\blindtext}
\end{mdframed}

\begin{mdframed}
    \textbf{Figure: Heatmap $T_\mathrm{eff}$ vs $\log{g}$ for a single line $2\times2$ panels for $\sigma$, $\gamma$, $A$, $\lambda$ }
\end{mdframed}

\begin{mdframed}
    \textbf{Anomalies in Heatmaps}
    \textcolor{lightgray}{\blindtext}
\end{mdframed}

\section{Mapping Line Parameters to Fundamental Properties}
\begin{mdframed}
    \textbf{Statement: A Bidirectonal Relation exists}
    - Goal: identify a functional form
    \textcolor{lightgray}{\blindtext}
\end{mdframed}

\begin{mdframed}
    \textbf{Figure: Flowchart-with-equations}
\end{mdframed}

\begin{mdframed}
    \textbf{Functional Form}
    - Describe functional form options and refinement procedures
    - List possibilities, Linreg, GPs \citep{2023ARA&A..61..329A}, NNs, etc.
    - We choose LSTSQ
    - Enumerate functional form
    - Adapting model complexity-- AIC
    \textcolor{lightgray}{\blindtext}
\end{mdframed}

\begin{mdframed}
    \textbf{Problem: missing lines}
    - Conceivable Solution 1: treat as NaNs and deal with sparsity
    - Conceivable Solution 2: increase model complexity
    \textcolor{lightgray}{\blindtext}
\end{mdframed}

\section{Performance evaluation}

\begin{mdframed}
    \textbf{Typical Line reconstruction performance}
    \textcolor{lightgray}{\blindtext}
\end{mdframed}

\begin{mdframed}
    \emph{stretch goal}\par
    \textbf{End-to-end PHOENIX grid replication and residual}
    - State the per-pixel residual
    \textcolor{lightgray}{\blindtext}
\end{mdframed}


\section{Discussion}
\begin{mdframed}
    \textbf{Revisiting Model Assumptions}

    \textcolor{lightgray}{\blindtext}
\end{mdframed}

\begin{mdframed}
    \textbf{Limitations}

    - Computational resources
    - Line profile inaccuracy
    - Surface functional form

    \textcolor{lightgray}{\blindtext}
\end{mdframed}


\begin{mdframed}
    \textbf{Conceivable Extensions}

    \textcolor{lightgray}{\blindtext}
\end{mdframed}


\pagebreak
\newpage

\begin{acknowledgments}
    \blindtext
\end{acknowledgments}


\software{}

\bibliography{ms}
\clearpage

\appendix
\section{Linear Regression}
\blindtext

\end{document}

