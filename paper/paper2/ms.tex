\documentclass[twocolumn]{aastex631}
\bibliographystyle{aasjournal}

\usepackage{amsmath}
\usepackage{blindtext}
% Temporary packages
\usepackage{mdframed}

\begin{document}
\title{Blas\'e3D}
\shorttitle{\emph{blas\'e3D}}

\author[0000-0002-2290-6810]{Sujay Shankar}
\author[0000-0002-4020-3457]{Michael Gully-Santiago}
\author[0000-0002-4404-0456]{Caroline V. Morley}
\affil{Department of Astronomy, The University of Texas at Austin, 2515 Speedway, Austin, TX 78712, USA}
\shortauthors{Shankar \& Gully-Santiago \& Morley}

\begin{abstract}
    \blindtext
\end{abstract}

\keywords{}


\section{Introduction}
\blindtext

\begin{mdframed}
    \textbf{Figure: Flowchart-with-words}
\end{mdframed}

\section{Emulating the PHOENIX grid}

\begin{mdframed}
    \textbf{The PHOENIX grid subset}

    \textcolor{lightgray}{\blindtext}
\end{mdframed}

\begin{mdframed}
    \textbf{Emulation with blas\'e}

    - Per-grid point optimization procedure
    - Pretrained model caching
    - Output storage

    \textcolor{lightgray}{\blindtext}
\end{mdframed}

\begin{mdframed}
    \textbf{Open Source Availability}

    \textcolor{lightgray}{\blindtext}
\end{mdframed}

\begin{mdframed}
    \textbf{Figure: Line density across the grid heatmap}
\end{mdframed}


\section{Line-by-Line Fundamental Stellar Properties}
\begin{mdframed}
    \textbf{Conceptual Illustration: Faceted Plot (Teff, Logg, Z) -> Line Profiles}
\end{mdframed}

\begin{mdframed}
    \textbf{Line Recognition}
    - Need: Identifying unique lines
    - Anticipated friction points: line centroids drift
    - Strategy: (Pre-shift centers)
    \textcolor{lightgray}{\blindtext}
\end{mdframed}

\begin{mdframed}
    \textbf{Line property bulk trends across the grid}
    \textcolor{lightgray}{\blindtext}
\end{mdframed}

\begin{mdframed}
    \textbf{Figure: Heatmap $T_\mathrm{eff}$ vs $\log{g}$ for a single line $2\times2$ panels for $\sigma$, $\gamma$, $A$, $\lambda$ }
\end{mdframed}

\begin{mdframed}
    \textbf{Anomalies in Heatmaps}
    \textcolor{lightgray}{\blindtext}
\end{mdframed}

\section{Mapping Line Parameters to Fundamental Properties}
\begin{mdframed}
    \textbf{Statement: A Bidirectonal Relation exists}
    - Goal: identify a functional form
    \textcolor{lightgray}{\blindtext}
\end{mdframed}

\begin{mdframed}
    \textbf{Figure: Flowchart-with-equations}
\end{mdframed}

\begin{mdframed}
    \textbf{Functional Form}
    - Describe functional form options and refinement procedures
    - List possibilities, Linreg, GPs \citep{2023ARA&A..61..329A}, NNs, etc.
    - We choose LSTSQ
    - Enumerate functional form
    - Adapting model complexity-- AIC
    \textcolor{lightgray}{\blindtext}
\end{mdframed}

\begin{mdframed}
    \textbf{Problem: missing lines}
    - Conceivable Solution 1: treat as NaNs and deal with sparsity
    - Conceivable Solution 2: increase model complexity
    \textcolor{lightgray}{\blindtext}
\end{mdframed}

\section{Performance evaluation}

\begin{mdframed}
    \textbf{Typical Line reconstruction performance}
    \textcolor{lightgray}{\blindtext}
\end{mdframed}

\begin{mdframed}
    \emph{stretch goal}\par
    \textbf{End-to-end PHOENIX grid replication and residual}
    - State the per-pixel residual
    \textcolor{lightgray}{\blindtext}
\end{mdframed}


\section{Discussion}
\begin{mdframed}
    \textbf{Revisiting Model Assumptions}

    \textcolor{lightgray}{\blindtext}
\end{mdframed}

\begin{mdframed}
    \textbf{Limitations}

    - Computational resources
    - Line profile inaccuracy
    - Surface functional form

    \textcolor{lightgray}{\blindtext}
\end{mdframed}


\begin{mdframed}
    \textbf{Conceivable Extensions}

    \textcolor{lightgray}{\blindtext}
\end{mdframed}


\pagebreak
\newpage

\begin{acknowledgments}
    \blindtext
\end{acknowledgments}


\software{}

\bibliography{ms}
\clearpage

\appendix
\section{Notation}


\begin{deluxetable}{cp{10cm}}
    \tabletypesize{\scriptsize}
    \tablecaption{Notation used in this paper\label{table2}}
    \tablehead{
        \colhead{Symbol} & \colhead{Meaning}
    }
    \startdata
    \hline
    \multicolumn{2}{c}{Spectra}\\
    \hline
    $\bm{\lambda}_S$ & Native wavelength coordinates of the precomputed stellar spectrum\\
    $\bm{\lambda}_T$ & Native wavelength coordinates of the telluric spectrum\\
    $\bm{\lambda}_D$ & Native wavelength coordinates of the data spectrum\\
    $\mathsf{S}_{\rm abs}$ & Flux values of the precomputed synthetic stellar spectral model $\bm{\lambda}_S$\\
    $\mathsf{B}$ & Blackbody of temperature $T_{\mathrm{eff}}$ to coarsely normalize $\mathsf{S}_{\rm native}$\\
    $\mathsf{P}$ & Smooth polynomial to refine continuum-normalization\\
    $\mathsf{S}$ & Continuum normalized augmentation of $\mathsf{S}_{\rm abs}$\\
    $\mathsf{T}$ & Transmission values of the precomputed synthetic telluric model \\
    $\mathsf{D}$ & The observed data spectrum flux values\\
    $\bm{\epsilon}$ & The estimated uncertainties in the data spectrum\\
    $\mathsf{S}_{\rm clone}$ & Evaluable and tunable cloned flux model of $\mathsf{S}$\\
    $\mathsf{T}_{\rm clone}$ & Evaluable and tunable cloned transmission model of $\mathsf{T}$\\
    $\mathsf{S}_{\rm ext}$ & An augmentation of $\mathsf{S}_{\rm clone}$ with $v\sin{i}$ convolution and $RV$ translation\\
    $\mathsf{M}_{\rm joint}$ & The joint stellar and telluric model: $\mathsf{S}_{\rm ext} \odot \mathsf{T}_{\rm clone}(\bm{\lambda}_S)$ \\
    $\mathsf{M}$ & Joint model convolved
    with instrumental kernel and resampled to $\bm{\lambda}_D$\\
    $\mathsf{R}$ & The residual spectrum between a pair of inputs, \emph{e.g.} $\mathsf{D} - \mathsf{M}$\\
    $\bm{v}$ & The spectral coordinate axis $\bm{\lambda}$ expressed as a velocity difference\\
    \hline
    \multicolumn{2}{c}{Line properties}\\
    \hline
    $\lambda_{\mathrm{c},j}$ & Line center position of the $j^{th}$ spectral line\\
    $a_j$ & Gaussian line profile amplitude of the $j^{th}$ spectral line \\
    $\sigma_j$ & Gaussian line profile scale of the $j^{th}$ spectral line\\
    $\gamma_j$ & Lorentzian line profile half width of the $j^{th}$ spectral line\\
    $\mathsf{V}_j$ & The Voigt profile of the $j^{th}$ spectral line \\
    $\bar{\bm{F}}$ & The dense $(N_{\rm lines} \times N_{x})$ matrix of all line fluxes stacked vertically \\
    $\hat{\bm{F}}$ & The sparse $(N_{\rm lines} \times N_{\rm sparse})$ matrix of all line fluxes stacked vertically \\
    $\zeta$ & The rotational broadening convolution kernel\\
    $g$ & The instrumental broadening convolution kernel, typically a Gaussian\\
    \hline
    \multicolumn{2}{c}{Scalars}\\
    \hline
    $N_{\rm lines}$ & Number of spectral lines \\
    $N_{x}$ & Number of pixel coordinates in the precomputed spectrum $\bm{\lambda}_x$\\
    $N_{\rm sparse}$ & Number of non-zero pixels computed in the sparse implementation\\
    $\pm \Delta \lambda_{\mathrm{buffer}}$ & Buffer exceeding the red and blue limits of the data spectrum\\
    $P_{\rm rom}$ & The prominence threshold of spectral lines to include in cloning \\
    $v\sin{i}$ & Rotational broadening for stellar inclination $i$ and equatorial velocity $v$\\
    $RV$ & Radial velocity of the star\\
    $R$ & Spectrograph resolving power $\lambda/\delta\lambda$\\
    $\mathcal{L}$ & The loss scalar, usually the sum of the squares of the residuals\\
    \hline
    \multicolumn{2}{c}{Operators}\\
    \hline
    $\resample \big[ \mathsf{F(\bm{\lambda}_x)} \big]$ & The resample operator, takes in a flux spectrum $\mathsf{F}$ evaluated at $\bm{\lambda}_x$ coordinates and returns the mean flux within the pixel boundaries of coordinate $\bm{\lambda}_z$\\
    $*$& The convolution operator\\
    $\odot$& \emph{Hadamard product}, an elementwise product of two same-length vectors\\
    \enddata
\end{deluxetable}

\end{document}

