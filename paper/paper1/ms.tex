\documentclass[modern]{aastex631}
\bibliographystyle{aasjournal}
\turnoffedit
\usepackage[caption=false]{subfig}
\usepackage{booktabs}
\usepackage{censor}
\usepackage{mathtools}  

\usepackage{outlines}
\usepackage{amsmath,bm}

\def\Teff{T_{\rm eff}}
\def\vsini{v\sin{i}}
\def\kmps{\mathrm{km}\;\mathrm{s}^{-1}}

\begin{document}
\shorttitle{blas\'e}
\shortauthors{Gully-Santiago}
\title{Transfer learning for echelle spectroscopy}

\author{Michael Gully-Santiago}
\affiliation{University of Texas at Austin Department of Astronomy}


\begin{abstract}

  We introduce blas\'e, a framework for transfer learning for high-grasp echelle spectroscopy.

\end{abstract}

\keywords{High resolution spectroscopy (2096)}

\section{Introduction}\label{sec:intro}

\subsection{Spectral fitting past and present}

Below is a table exploring recent techniques for spectral fitting.

\movetabledown=13mm
\begin{rotatetable}
  \begin{deluxetable}{chc}
    \tablecaption{Annotated bibliography for intro\label{table1}}
    \tablehead{
      \colhead{Reference} & \nocolhead{two} & \colhead{Key idea}
    }
    \startdata
    \citet{2015PhDT........82P} & - & \texttt{specmatch-syn} Synthetic spectral template matching \\
    \citet{czekala15} & - & \texttt{starfish} robust spectral inference on precomputed models \\
    \citet{2016ApJS..223....8C} & - & \texttt{sick} Spectral Inference Crank, similar to \texttt{Starfish}\\
    \citet{2017ApJ...836..200G} & - & Modular extensions to \texttt{starfish}: starspots \\
    \citet{2018ApJ...862...85G} & - & Modular extensions to \texttt{starfish}: veiling \\
    \citet{2017ApJ...836...77Y} & - & \texttt{specmatch-emp} Empirical template matching for visible\\
    \citet{exoplanet:astropy18} & - & \texttt{astropy} General purpose tools for spectral manipulation\\
    \citet{2019AJ....158..164B} & - & \texttt{wobble}: data driven stellar and telluric models
    \enddata
  \end{deluxetable}

\end{rotatetable}



\subsection{Seeking physical self-consistency and high predictive accuracy}
Ideally we seek models that are both physically self-consistent and highly accurate when compared to data.  We want tunable models that can be evaluated quickly for use in likelihood calculations.  We want posterior inference to accurately explore correlations among parameters.  We want unbiased, noise-free templates for use in cross-correlation spectroscopy.  We want accuracy over a large spectral grasp: simultaneously wide bandwidth and high spectral resolution.  We want high out-of-band prediction accuracy.  Ideally we also want precision and accuracy in the separation of affiliated telluric and instrumental signals.

These goals form the backbone of spectral calibration, an increasingly valuable specialty as the deficits in our models become intolerable with greater data quantity and quality.

\subsection{Current limitations to our spectral models}

Several practical barriers limit the accuracy of spectral models.  First and foremost, real stars are usually more complicated than our simplified models of them.  Real spectra often vary over more dimensions than specified.  Conspicuous examples of these hidden variables can be found in protostars: starspots, accretion veiling, dust extinction, and magnetic Zeeman splitting.  Jointly modeling all of these phenomena alongside the intrinsic stellar photosphere and  High latent dimensionality: Many effects matter at the precision level of the data...

Second, Degeneracy among parameters

\begin{itemize}
  \item Underlying atomic and molecular data may be wrong or approximate or missing
  \item Underlying radiative transfer may be inaccurate or approximate (T-P profile / T-Tau)
  \item Time is a dimension: Star is changing
  \item Stellar surface inhomogeneities become important: Doppler Imaging and limb darkening
  \item We do not know how to accurately parameterize some dimensions
\end{itemize}

\subsection{High dynamic range is both a blessing and a curse:}
\begin{itemize}
  \item Feeding back locations of model imperfections becomes unwieldy for many lines
  \item Many lines essentially becomes a book-keeping and continuum assignment problem
  \item Computational problem
  \item Some lines have large line wings, blurring continuum and line
  \item Line blanketing has almost no continuum
\end{itemize}

\subsection{This paper: towards semi-empirical models}
\begin{itemize}
  \item So in practice, tradeoffs between model flexibility and physical self-consistency
  \item Semi-empirical models may be a middle ground: informed from models, but revised with data
  \item The radiative transfer step is computationally expensive, so we want to avoid redoing that
  \item Instead focus on pre-computed models that have that wisdom baked-in
  \item We are essentially compressing pre-computed models into evaluable, interpretable models
  \item Transfer learning is one plausible approach: pre-train on pre-computed models, transfer to data
  \item Neural network frameworks make this possible with autodiff
  \item We use the autodiff and GPU acceleration, but not the neural architectures
  \item We do not use the neural architectures because we have good understand for how a spectrum is generated
  \item Instead of tuning weights of a neural network, we tune the atomic and molecular properties of a spectrum
\end{itemize}


\section{Methodology}

\subsection{Overall Architecture and Design Choices}

We start with a high resolution pre-computed synthetic stellar model spectrum, $\mathsf{S}(\bm{\lambda})$.  The procedure is largely agnostic to the exact details of how this spectrum was made.  For the purposes of this paper, we will showcase examples from two well-known families of precomputed synthetic spectra: \texttt{PHOENIX} \citep{husser13} for stellar spectra $(T_{\mathrm{eff}}\in [2300, 10000]\;K)$ and \texttt{Sonora} \citep{2021ApJ...920...85M} for brown dwarf spectra $(T_{\mathrm{eff}}\in [1000, 2300]\;K)$.  We place the following demands on the the precomputed spectra.  They should have sporadic regions of discernable continuum devoid of lines, and the continuum must vary smoothly in wavelength.  The spectral lines or pseudo-lines should be resolved, and not sub-sampled.  We suspect most stellar spectra meet these criteria, with the exception of the coolest M dwarfs and brown dwarfs.  The method can hypothetically handle spectra without discernable continua, but we suspect some modifications to the preprocessing steps would be needed.  We truncate the red and blue limits of the precomputed synthetic spectrum to match a high-bandwidth echelle spectrograph, extended with a buffer at the edges of size $\pm \Delta \lambda_{\mathrm{buffer}}$, chosen to account for plausible radial velocity and rotational broadening of real stars.  A generous buffer of $v \sin{i} < 500 \;\kmps$ and $|RV|<500 \;\kmps$ yields a typical buffer of about 30~\AA.

The choice of limiting the bandwidth to a region-of-interest around a single echelle spectrograph bandwidth stems from computational constraints.  In principle there is no fundamental limit to the bandwidth one could clone with the method presented here, up to and including the entire precomputed synthetic spectral model bandwidth.  We adopt the exact native wavelength sampling with no smoothing or interpolation, yielding a wavelength vector $\bm{\lambda}_s$ with length $N_\mathrm{s}$ equal to the number of pixels within the extents of our region of interest including the buffers.

At this stage we have the choice of whether to work in linear or log scale flux units.  Adopting the log of the flux would ensure that the cloned model possesses only positive flux values, a desireable trait of any physical spectral model.  We chose instead to stick with native linear flux scaling because of the availability of convenient approximations to astrophysically relevant line profiles in the linear scaling domain.  We normalize the spectrum by its median value since we do not intend to apply these methods to absolutely flux calibrated spectra.

\subsection{Initialization}

We initialize the cloned model with a series of preprocessing steps.  We divide the entire spectrum by a black body of the same effective temperature $T_{\mathrm{eff}}$ as the model template.  The resulting signal typically possesses a smooth continuum variation.  An additional continuum flattening step ensures that subsequent spectral line finding steps get applied uniformly.  Here, we identify the continuum regions as the most prominent peaks in the specturm separated by a pixel distance of at least $15000-90000$ pixels from adjacent peaks, yielding between 60 and 5 points.  We then fit an $n_{\mathrm{poly}} = 3^{rd}-5^{th}$ order polynomial to these peaks, evaluate the polynomial on the entire bandwidth, and divide the spectrum by this trend.  Visual inspection of this continuum-flattened spectrum typically possesses variations at the $<0.2\%$ level.  Tuning the pixel distance and polynomial order can lower the undulations to $<0.05\%$ level for some input spectra.  The high-pass filtering step should be set to capture the genuine spectral shape, without over-fitting broad line-wings such as those in deep Hydrogen and sodium lines.  We have experimentally found it helpful to hand-tune models of exceptionally broad and non-standard line-wings with an optional pre-processing step, not described here.

We emphasize that a recreation of the unvarnished input spectrum can be obtained by multiplying the continuum-flattened signal by the ``perturbed black body'', $\mathsf{B}(\bm{\lambda}_s)$, that symbolizes the black body modulated by continuum opacity or broad-band radiative transfer effects.  This smooth spectrum may be useful for applications that need to keep track of broad-band flux, such as low resolution spectrographs, or regions with molecular bandheads.  The ``perturbed black body'' continuum model $\mathsf{B}$ contains  $n_{\mathrm{poly}}+1$ fixed lookup parameters, plus the input $T_{\mathrm{eff}}$.


Next we identify the spectral lines.  We apply a local-minimum-finding algorithm on the spectrum by defining a prominence threshold $P_{rom} \in (0.005, 0.1)$.  This threshold dictates the number of lines that will be modeled: a lower prominence finds more, weaker lines, and a larger prominence finds fewer, deeper lines.  The prominence algorithm successfully finds lines that reside on top of broad line wings, or unresolved bandheads provided that the individual lines exceed the prominence threshold in their local region.  The number of lines $N_{\mathrm{lines}}$ depends on the bandwidth, prominence, and the intrinsic properties of the input spectrum, principally effective temperature and metallicity.

For this paper, we illustrate examples for two \'echelle spectrographs with particularly large spectral grasp: the Habitable Zone Planet Finder \citep[HPF,][]{2014SPIE.9147E..1GM} on the Hobby-Eberly Telescope at McDonald Observatory in Fort Davis, Texas; and the Immersion Grating Infrared Spectrograph \citep[IGRINS,][]{park14}
currently on the Gemini South Telescope on Cerro Pach\'on in Chile.  The $R=55,000$ HPF has a native bandwidth of $x-y$, that we expand to $x-y^*$ including the edge buffers.  IGRINS has two cameras for $H$ and $K$ band, with the combined spectrum spanning $z-y$ including the edge buffers and the region in-between the two cameras, all at a resolving power of $(R=45,000)$.  The spectrograph acquisition, reduction, and post-processing steps yield data $\mathsf{D}(\bm{\lambda}_{d})$, where $\bm{\lambda}_{d}$ is the wavelength vector at the instrumental resolution and sampling of each instrument, generally much coarser than the resolution and sampling grid of the precomputed synthetic spectra.  The data wavelength vector may also contains gaps between \'echelle orders, whereas the precomputed grid is usually contiguous.  HPF may have up to $2048\times28=57,344$ pixels, and IGRINS has typically about $75,000$ pixels, after common truncation of noisy edge pixels.

Figure XX shows how the number of detected lines $N_{\mathrm{lines}}$ scales with effective temperature and prominence threshold $P_{rom}$ for the \texttt{PHOENIX} grid, truncated to the bandwidths-plus-buffers for HPF and IGRINS.  The HPF-truncated spectra have $N_s=XX$ and the IGRINS-truncated spectra have $N_s=XX$.

So far we have only one piece of information about the peaks: their location.  Next, we derive coarse properties about each detected peak: its amplitude and width, again using the prominence algorithms implemented in \texttt{scipy} \citep{2020SciPy-NMeth}.

There does not exist a general-purpose, single-shot algorithm for obtaining the lineshape in the presence of overlapping spectral lines: where do the wings of one line begin and the wings of another adjacent line end?  We therefore do not attempt to determine anything about the lineshape at this stage, and instead assume that the lines resemble a Voigt profile, with a guess width about equally split between Lorentzian and Gaussian.

\subsection{The \emph{blas\'e} clone model}

We have now arrived at the \emph{blas\'e} clone model $\mathsf{M}(\bf{\lambda}_s)$ for a flattened synthetic spectrum $\mathsf{S}$: it is the cumulative product of transmission through the sea of all overlapping spectral lines:

\begin{eqnarray}
  \mathsf{M}(\bm{\lambda}_s) = {\displaystyle \prod_{j=1}^{N_{\mathrm{lines}}}  (1-a_j \mathsf{V}(\bm{\lambda}_s-\lambda_{c,j}, \sigma_j, \gamma_j) )} \label{equation1}
\end{eqnarray}

where $\mathsf{V}(\lambda, \sigma_j, \gamma_j)$ is the normalized Voigt profile with Gaussian standard deviation $\sigma$, Lorentzian half-width $\gamma$, at line center position $\lambda_c$, for the $j^{th}$ spectral line.  The amplitude $a$ is always expected to be positive for absorption lines, and is equal to the area under the curve of the line since the Voigt profile is normalized.

The model evaluated with its coarse initial values would have terrible performance: it would only vaguely resemble the synthetic spectral model, with up to $\pm 50\%$ undulations from the inexact assignment of widths, lineshapes, and amplitudes.  Instead, we tune the parameters of the model, starting from these coarse initial values.  This model has $N_{\mathrm{lines}}\times 3$ free parameters, where the center wavelength is held fixed and the amplitude, width, and lineshape are allowed to vary.  We minimize a scalar ``goodness-of-fit'' metric, \emph{aka} loss scalar $\mathcal{L}$, chosen as the mean squared error (MSE), which is proportional to $\chi^2$, the sum of the squares of the residual vector $\mathsf{R} \equiv \mathsf{S}-\mathsf{M}$ but has no notion of per-pixel noise, since the precomputed synthetic spectrum has no uncertainty:

\begin{eqnarray}
  \mathcal{L} = \sum_i^{N_s} (S_i - M_i)^2 = \mathsf{R^\intercal}\cdot \mathsf{R}
\end{eqnarray}


As seen in Figure XX, the number of lines can exceed $XX$, meaning the model has over $XX \times 3 = ZZ$ free parameters.  Fitting that large number of parameters is difficult with conventional optimizers, which struggle to converge after hundreds of parameters\authorcomment1{Citation to limits of optimizers}.

Here we employ a variant of Stochastic Gradient Descent (SGD), an optimization technique that can scale to a virtually unlimited number of parameters\authorcomment1{provide a citation}.  This technique computes the derivative of the loss scalar with respect to each of the parameters, the Jacobian: $(\frac{\partial \mathcal{L}}{\partial a_j}, \frac{\partial \mathcal{L}}{\partial \sigma_j}, \frac{\partial \mathcal{L}}{\partial \gamma_j})$.  The Jacobian indicates how the MSE would decrease with a change in the parameter-of-interest, or put simply which-way and by-how-much you have to change the line properties to get a better fit.

The optimizer updates the $a_j, \sigma_j, \gamma_j$ parameters by a small fraction of the Jacobian---called the learning rate LR---towards the direction that would improve the fit, for all $N_{\mathrm{lines}} \times 3$ parameters simultaneously.  The Jacobian is calculated behind-the-scenes with automatic differentiation\authorcomment1{Citation} implemented as the so-called backwards-propagation algorithm or simply ``back-prop''.  We choose the \texttt{PyTorch} framework that computes these Jacobians efficiently for all of the mathematical primatives in our \texttt{blas\'e} implementation.

In principle we could compute the Jacobians analytically---since the partial derivatives of the Voigt function are known in closed-form---circumventing the vendor lock-in and memory overhead of an autodiff implementation with PyTorch.  Futher, an autodiff-aware implementation of the \emph{celerit\'e} algorithm\authorcomment1{add DFM citation} does not yet exist in PyTorch, constricting our choices for simultaneous continuum fitting.  In practice, PyTorch also offers hardware acceleration (on GPU or TPU)\authorcomment1{citation needed}, mature support for a range of optimizers, and other perks that motivated its choice over analytic implementations or other machine learning frameworks.  The GPyTorch framework\authorcomment1{add citation} may offer a workaround to the absence of a \emph{celerit\'e} implementation.  Still, the performance of an analytic Jacobian implementation over the autodiff route presented here offers an interesting avenue of future research.

\subsection{GPU and Autodiff specific considerations}
We make a few tweaks to the implementation for numerical purposes.  First, we want all the parameters to be positive, forbidding negative amplitudes and negative widths.  We therefore tune the natural log of the parameters, and exponentiate them before inclusion in Equation \ref{equation1}.  Second, we found through iterative experimentation that the initialization amplitudes and widths were systematically shifted from the optimized values.  We built-in scalar tweaks to the initialization amplitudes, Gaussian width, and Lorentzian width, which dramatically accelerated the optimization step.

We set the \texttt{requires\_grad=True} property for any Torch tensor that we want to vary.  This allows us to easily explore whether, say, allowing the $\lambda_c$ parameter to vary significantly improves the fit.  It also allows us to tune or freeze an experimental wavelength-dependent pre-factor to Equation \ref{equation1} to correct for lingering imperfections in the otherwise-fixed continuum flattening procedure.

The computational bottleneck occurs at the evaluation of Equation \ref{equation1}, which includes the assembly of a $N_{\mathrm{lines}}\times N_{s}$ matrix $\bm{\bar{V}}$ assembled by stacking each Voigt profile $\mathsf{V}_j(\bm{\lambda}_s)$ on top of each other:

\begin{equation}
  \begin{pmatrix}
    \mathsf{V}_1(\bm{\lambda}_s)                    & \\
    \mathsf{V}_2(\bm{\lambda}_s)                    & \\
    \vdots                                          & \\
    \mathsf{V}_{N_{\mathrm{lines}}}(\bm{\lambda}_s) &
  \end{pmatrix}
\end{equation}

$\mathsf{V}_j(\bm{\lambda}_s)$ matrix  .





\begin{outline}
  \1 Total number of parameters
  \2 For the lines =  N-lines x 4 (lambda center, fwhm L, sigma G, Amplitude)
  \2 For the continuum = Nth order polynomial = N+1 parameters, or Gaussian Process = 2 hyperparameters
  \1 We assemble the product of all lines and all continuum sources, this is the computational bottleneck
  \1 This becomes a matrix of size N-lines x N-wavelength points
  \1 We take the element-wise product over the lines axis, contracting to 1 x N-wavelength points
  \1 For N-lines = X and N-wavelength-points = Y, we expect X*Y*k primative operations.
  \1 For N-lines = X and N-wavelength-points = Y, we expect X*Y*32 bits of memory => XX GB
  \1 NVIDIA GPUs can handle 8 - 40 GB of memory depending on the model
\end{outline}

\subsection{Minibatches}
\begin{outline}
  \1 In the parlance of Machine Learning, we have a big model and small data
  \1 Really our model is too big: most of the entries are zero, so we should use a sparse implementation
  \1 If the model is too big to fit in memory, we have to train in mini-batches
  \1 We hand-in some fraction f ~ 1/20th of the entire wavelength range as a mini-batch
  \1 The model is only evaluated at that 5\% of pixels,  $N_{lines} \times N_{pix} \cdot 0.05$)
  \1 Technically all lines are updated
  \1 In practice, only lines that are sufficiently informed by those pixels will get large updates
  \1 Causes some stochasticity--- Stochastic Gradient Descent "SGD"
  \1 Analogous but different from the notion of minibatch in ML: not a distinct axis, like $N_{images}$
  \1 We use random minibatches
\end{outline}

\subsection{Optimization and training}
\begin{outline}
  \1 Run the training:
  \1 Set the optimizer: Adam with a LR=0.004
  \1 Set the loss function: Chi-squared with no per-pixel uncertainty
  \1 Set the "mini-batch" fraction f~1-5%
  \1 Choice of epochs or steps $N_steps = 300-10000$
  \1 The training process sees 20 minibatches per epoch, sees $N_pixels$, with some repeats some misses
  \1 Monitor with Tensorboard: see the improvement
  \1 The training takes X minutes on an RTX2070 GPU with PyTorch v1.11.x, CUDA v11.x, and Intel X CPU
  \1 We save the model parameters, refer to them collectively as the pre-trained model
\end{outline}


\section{Cloning Performance}

\subsection{Results and scaling}

\begin{outline}
  \1 We compute the residual of native PHOENIX minus cloned model
  \1 We see X percent residuals at native resolution
  \1 We smooth the residual to HPF instrumental resolution and see Y percent residuals
  \1 The main residuals come from missing line opacity due to our prominence threshold:
  \1 Including smaller prominence lines means smaller residuals, but more lines and higher computational cost
\end{outline}

\subsection{Line data clustering and conspicuous flaws}
\begin{outline}
  \1 We see some lines that devolve into missing continuum opacity
  \1 Especially acute near broad line wings if you don't handle the non-standard line shapes
  \1 How do the lineshapes cluster in $\sigma_G$ vs $FWHM_L$?
  \1 How do the lineshapes cluster in Amplitude vs FWHM? Can some lines be rejected?
  \1 Does the pre-trained model round-trip effectively (model saved at end of training = model from loaded params)?
\end{outline}


\section{Refining the model from data with transfer learning}
\begin{outline}
  \1 Augment the pre-trained model with extrinsic parameters:
  \2 vsini and RV
  \1 How to re-sample into the data space?
  \1 Option 1: Convolve the Pseudo-Voigt with a Instrumenal Gaussian
  \2 The convolution of a Gaussian with a Pseudo-Voigt is simply a new Pseudo-Voigt!
  \2 Simply evaluate on a new coarse wavelength grid!  Saves computation!
  \1 Option 2: Numerically convolve output on a fine grid and sum pixels
  \2 Computationally wasteful
  \2 Requires generating full bandwidth spectrum at native resolution
  \1 Both options should give the same answer (right?)
  \2 Need to double-check commutivity of convolution and element-wise product...
  \1 Ideally you need wavelength-dependent resolution, not just a single stock resolving power
  \1 Need accurate per-pixel uncertainties
  \1 We have to apply some amount of regularization to the pre-trained model
  \2 Regularization tunes how much to trust the data versus the model in the semi-empirical model
  \2 No Regularization means likely to overfit noise in the data
  \2 Too much regularization resembles a static (fixed) PHOENIX model
\end{outline}

\section{Performance of transfer learning}
\begin{outline}
  \1 What is the residual level with bare PHOENIX (i.e. tuning continuum only, not lines)?
  \1 What is the residual level with no regularization (i.e. tune all lines and continuum)?
  \2 Should be near-zero except for missing lines
  \2  What should we do about missing lines?
  \1 What is the residual level with modest regularization?  What is the typical change to the line?
  \2 Plot of cloned FWHM versus FWHM transferred
  \2 The pseudo-voigt line properties can be analytically integrated to give an equivalent width
  \2 Plot of cloned EW (before) versus transfer EW (after)
\end{outline}

\section{Discussion}

\subsection{Lineshapes choice and tradeoffs}
\begin{outline}
  \1 Faced with a question: what lineshape to use
  \2 To First order the lines are Gaussian or Lorentzian
  \2 To Second order the lines are Voigt Profiles
  \2 In detail the lines are the result of radiative transfer that can smear the Voigt profile
  \2 In practice the high-res lines get convolved with instrumental profiles, so details don't matter too much
  \1 Computational tradeoff of Lorentzian/Gaussian versus Pseudo-Voigt versus Voigt
  \2 Additional Implementation challenge: Fadeeva function not implemented in PyTorch
  \1 We choose a pseudo-Voigt as a balance between computation and adequate accuracy
\end{outline}

\subsection{Questioning the assumptions of the method}

\begin{outline}
  \1 Is PseudoVoigt Adequate?
  \1 What performance level is adequate? (depends on your application)
  \1 What about PRV applications
  \1 What about the choice of minibatch size?
  \1 How would a sparse tensor change the computational performance?
  \1 Are we using the GPU effectively?  Are we memory bandwidth limited?
  \1 Should we update the line center positions?
\end{outline}

\subsection{Limitations}
\begin{outline}
  \1 What to do when continuum is absent even in at the native resolution of precomputed synthetic model?
  \1 Relevant to brown dwarfs and possibly M dwarfs molecular bands
  \1 Bandwidth limitations
  \1 Interplay with tellurics
\end{outline}

\section{Conclusions}
More placeholder text...


\

\begin{acknowledgements}
  The author acknowledges the Texas Advanced Computing Center (TACC, \url{http://www.tacc.utexas.edu}) at The University of Texas at Austin for providing HPC resources that have contributed to the research results reported within this paper.
\end{acknowledgements}

\clearpage


\facilities{HET (HPF)}

\software{  pandas \citep{mckinney10},
  matplotlib \citep{hunter07},
  astropy \citep{exoplanet:astropy13,exoplanet:astropy18},
  exoplanet \citep{exoplanet:joss}, %celerite?
  numpy \citep{harris2020array},
  scipy \citep{2020SciPy-NMeth},
  ipython \citep{perez07},
  starfish \citep{czekala15},
  seaborn \citep{Waskom2021},
  pytorch \citep{2019arXiv191201703P}}


\bibliography{ms}


\clearpage

\appendix
\restartappendixnumbering

\section{Autodiff themes} \label{appendix:tools}

Here are some more details about autodiff


% Table
\begin{deluxetable}{cc}
  \tablecaption{Notation used in this paper\label{table2}}
  \tablehead{
    \colhead{Symbol} & \nocolhead{Meaning}
  }
  \startdata
  $N_{pix}$ & Number of pixels in the precomputed spectrum \\
  $N_{lines}$ & Number of spectral lines \\
  $\sigma_G$ & Gaussian line profile scale and standard deviation \\
  $\gamma_L$ & Lorentzian line profile half width\\
  $\pm \Delta \lambda_{\mathrm{buffer}}$ & Buffer exceeding the red and blue limits of data spectrum
  \enddata
\end{deluxetable}

\end{document}
