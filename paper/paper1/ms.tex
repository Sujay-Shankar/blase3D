\documentclass[modern]{aastex631}
\bibliographystyle{aasjournal}
\turnoffedit
\usepackage[caption=false]{subfig}
\usepackage{booktabs}
\usepackage{censor}

\def\Teff{T_{\rm eff}}
\def\vsini{v\sin{i}}
\def\kmps{\mathrm{km}\;\mathrm{s}^{-1}}

\begin{document}
\shorttitle{blas\'e}
\shortauthors{Gully-Santiago}
\title{Transfer learning for echelle spectroscopy}

\author{Michael Gully-Santiago}
\affiliation{University of Texas at Austin Department of Astronomy}


\begin{abstract}

  We introduce blas\'e, a framework for transfer learning for high-grasp echelle spectroscopy.

\end{abstract}

\keywords{High resolution spectroscopy (2096)}

\section{Introduction}\label{sec:intro}

Here is an annotated bibliography.

\begin{deluxetable}{chc}
  \tablecaption{Annotated bibliography for intro\label{table1}}
  \tablehead{
    \colhead{Reference} & \nocolhead{two} & \colhead{Key idea}
  }
  \startdata
  \citet{2017ApJ...836..200G} & - & Extensions to Starfish
  \enddata
\end{deluxetable}

\newpage


\section{Methodology}

TBD

\section{Conclusions}

More placeholder text...


\begin{acknowledgements}
  The author acknowledges the Texas Advanced Computing Center (TACC, \url{http://www.tacc.utexas.edu}) at The University of Texas at Austin for providing HPC resources that have contributed to the research results reported within this paper.
\end{acknowledgements}

\clearpage


\facilities{HET (HPF)}

\software{  pandas \citep{mckinney10},
  matplotlib \citep{hunter07},
  astropy \citep{exoplanet:astropy13,exoplanet:astropy18},
  exoplanet \citep{exoplanet:joss}, %celerite
  numpy \citep{harris2020array},
  scipy \citep{2020SciPy-NMeth},
  ipython \citep{perez07},
  starfish \citep{czekala15},
  seaborn \citep{Waskom2021},
  pytorch \citep{2019arXiv191201703P}}


\bibliography{ms}


\clearpage

\appendix
\restartappendixnumbering

\section{Autodiff themes} \label{appendix:tools}

Here are some more details about autodiff
\end{document}
